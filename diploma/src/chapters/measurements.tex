\chapter{Statistici si Rezultate Obținute}


\section{Statistici Corpus Obținut cu Microsoft Academic Search}

În această secțiune vom pune în evidență statisticile corpusului obținut folosind Microsoft Academic Search. Este bine cunoscut faptul că \textit{Vocabularul} $V$ și numărul de cuvinte $N$ sunt reprezentative pentru un corpus.

\index{Corpus}
\index{Vocabular}

\textit{Vocabularul} reprezintă numărul de cuvinte unice pe care le conține un text.

\index{Științe Sociale}

\subsection{Statistici Documente PDF Download-ate}

În \labelindexref{Tabelul}{table:downloaded-pdf-summary}, avem rezultatele procesului de download pentru domeniul Științe Sociale. Putem observa că aproximativ 10\% din publicațiile găsite reprezintă documente PDF ce au putut fi efectiv download-ate. Ulterior, vom vedea că acest corpus este totuși absolut suficient pentru ceea ce avem noi nevoie.

\begin{center}
\begin{table}[htb]
  \caption{Rezultatele download-ului documentelor din Științe Sociale}
  \begin{tabular}{|l|r|}
    \hline
    Număr total autori & 99 \\
    \hline
    Număr total publicații & 8498 \\
     \hline
    Număr total PDF-uri download-ate & 962  \\
     \hline
  \end{tabular}
  \label{table:downloaded-pdf-summary}
\end{table}
\end{center}

\subsection{Statistici Cuvinte din Corpus}

După ce am download-at documentele, am folosit un parser de PDF, Apache Tika, pentru a extrage textul din ele. Apoi, am efectuat mai multe prelucrări asupra textului. Vor fi descirse ulterior.

Documentele text obținute au fost analizate pentru a determina dimensiunea vocabuluarului $|V|$ și numărul total de cuvinte.

Mai întâi, folosit câteva programe drăguțe din \texttt{Unix}, precum \texttt{tr}, \texttt{sort} și \texttt{wc}. Codul poate fi văzut în \labelindexref{Listingul}{lst:vocabulary-size-bash}, Ele au raportat următoarele date:

\begin{center}
\begin{table}[htb]
  \caption{Dimensiunea vocabularului și numărul de cuvinte ale corpusului folosind programe Unix}
  \begin{tabular}{|l|r|}
    \hline
    Dimensiune Vocabular |V| & 122 457\\
    \hline
    Număr total cuvinte & 9 249 223 \\
     \hline
  \end{tabular}
  \label{table:vocabulary-size-bash}
\end{table}
\end{center}


\lstset{language=bash}
\lstset{caption=Comenzile Unix folosite pentru determinarea vocabularului, label=lst:vocabulary-size-bash}
\lstinputlisting{src/code/build/vocabulary-size-bash.txt}

Este suprinzător numărul mare de cuvinte și vom vedea și de ce. În \labelindexref{Listingul}{lst:top-words-bash}, avem cele mai utilizate cuvinte din corpusul obținut. Primele cuvinte "the", "of", "and" era de așteptat să apară primele, deoarece sunt foarte folosite în limba engleză. În această statistică nu am ținut cont de capitalizare, de exemplu, "The" și "the" același lucru. Dar observăm că apar foarte des cuvinte \textit{token-uri} care nu sunt cuvinte din limba engleză, cum ar fi "s", "m", sau "e". Acest lucru este cauzat de faptul ce în engleză, posesia este construită cu John's, iar noi am considerat \textit{token-uri} separate "John" și "s". De aceea "s" apare de peste 50 000 de ori.

\lstset{language=make}
\lstset{caption=Cele mai folosite cuvinte din limba engleză pentru corpusul obținut folosind programe Unix, label=lst:top-words-bash}
\lstinputlisting{src/code/build/top-words-bash.txt}

Observând acest lucru, am decis să implementez în Java un algoritm mai robust pentru a determina statistici referitoare la corpus. Programele din Unix au oferit doar rezultate orientative.

Astfel, am implementat în Java un algoritm mai robust pentru a calcula dimensiunea vocabuluarului. Rezultatele pot fi observate în \labelindexref{Listingul}{lst:top-words-java}.

Folosind Java, am obținut următoarele dimensiuni ale corpusului, prezentate în \labelindexref{Tabelul}{table:vocabulary-size-java}:

\begin{center}
\begin{table}[htb]
  \caption{Dimensiunea vocabularului și numărul de cuvinte ale corpusului folosind Java}
  \begin{tabular}{|l|r|}
    \hline
    Dimensiune Vocabular |V| & 110 834\\
    \hline
    Număr total cuvinte & 9 703 480 \\
     \hline
  \end{tabular}
  \label{table:vocabulary-size-java}
\end{table}
\end{center}

\lstset{language=make}
\lstset{caption=Cele mai folosite cuvinte din limba engleză pentru corpusul obținut folosind Java, label=lst:top-words-java}
\lstinputlisting{src/code/build/top-words-java.txt}

De asemenea, am ignorat scrierea cu majusculă. Și aici "The" și "the" reprezintă același cuvânt.

\begin{description}
	\item[O primă observație] ce se poate face este că dimensiunile vocabularului și ale numărului total de cuvinte raportate de cele două metode de calcul sunt apropiate, de același ordin de mărime. Avem un \textbf{vocabular} de aproximativ \textbf{110 000 de cuvinte distincte} și în un\textbf{ număr total de peste 9 milioane de cuvinte}.
	\item [O a doua observație] ce se poate face este ca în statisticile raportate de Java au dispărut \textit{token-urile} bizare precum "s", "e", "m", "d" sau "onghaul".
\end{description}

\section{Statistici Cuvinte din Corpusul Adnotat}

În urma procesului de corectare manuală a adnotării, au rezultat 216 documente text. La fel ca întreg corpusul și aceste documente text au fost analizate pentru a determina dimensiunea vocabularuluiui $|V|$, dar și numărul total de cuvinte. Am analizat subsetul adnotat din corpus folosind atât programe din \texttt{Unix}, cât și modulul de statistici din Java.


\begin{center}
\begin{table}[htb]
  \caption{Dimensiunea vocabularului și numărul de cuvinte ale setului adnotat din corpus folosind programe Unix}
  \begin{tabular}{|l|r|}
    \hline
    Dimensiune Vocabular |V| & 9 556\\
    \hline
    Număr total cuvinte & 109 949 \\
     \hline
  \end{tabular}
  \label{table:annotated-corpus-vocab-size}
\end{table}
\end{center}

În labelindexref{Listingul}{lst:top-words-bash-annotated} este sunt evidențiate cele utilizate cuvinte din corpusul adnotat potrivit instrumentelor din Unix.

\lstset{language=make}
\lstset{caption=Cele mai folosite cuvinte din limba engleză pentru subsetul adnotat din corpus folosind programe Unix, label=lst:top-words-bash-annotated}
\lstinputlisting{src/code/build/top-words-bash-annotated.txt}

De asemenea, am generat statistici folosind programul scris în Java. Am obținut următoarele dimensiuni ale subsetului adnotat manual din corpus, prezentate în \labelindexref{Tabelul}{table:vocabulary-size-java-annotated}:

\begin{center}
\begin{table}[htb]
  \caption{Dimensiunea vocabularului și numărul de cuvinte ale subsetului adnotat manual din corpus folosind Java}
  \begin{tabular}{|l|r|}
    \hline
    Dimensiune Vocabular |V| & 8 630\\
    \hline
    Număr total cuvinte & 113 601 \\
     \hline
  \end{tabular}
  \label{table:vocabulary-size-java-annotated}
\end{table}
\end{center}

\lstset{language=make}
\lstset{caption=Cele mai folosite cuvinte din limba engleză pentru subsetul adnotat din corpus obținut folosind Java, label=lst:top-words-java-annotated}
\lstinputlisting{src/code/build/top-words-java-annotated.txt}

Observăm ca valorile raportate de cele doua programe sunt apropiate în limita unei marje de eroare, de aproximativ 9 000 de cuvinte unice în vocabular și de aproximativ 110 000 de cuvinte în total. De asmemenea, topul celor mai frecvente cuvinte este asemănător atât la corpusul mare cât și la subsetul adnotat.

\textbf{În concluzie}, eu am adnotat 216 din peste 20 000 de documente, însemnând cam 1\% din total. Numărul total de cuvinte a scăzut de la peste 9 milioane de cuvinte la puțin peste 110,000 de cuvinte, deci aproximativ 1.15\% din total. Numărul de cuvinte a scăzut proporțional cu numărul de documente. Pe de altă parte vocabularul s-a redus de la 110,000 de cuvinte la 9,000 de cuvinte. Aceasta reprzintă 8.1\% din vocabularul inițial. Reducerea nu a fost liniară, ceea ce era de așteptat.

\section{Statistici Entități din Corpusul Adnotat}

Am analizat repartiția entităților pe corpusul adnotat manual. 


\begin{center}
\begin{table}[htb]
  \caption{Dimensiunea vocabularului și numărul de cuvinte ale subsetului adnotat manual din corpus folosind Java}
  \begin{tabular}{|l|r|}
  \hline
   Categorie Entitate & Număr apariții\\
   \hline
   
   TIME	&	3089	\\
   LOCATION	&	2549	\\
   SET	&	1988	\\
   ORGANIZATION	&	1007	\\
   PERSON	&	912	\\
   MONEY	&	561	\\
   PERCENT	&	381	\\
   MISC	&	355	\\
   DATE	&	248	\\
   NATIONALITY	&	237	\\
   DURATION	&	73	\\
   ORDINAL	&	65	\\
   NUMBER	&	29	\\
   \hline
   \textbf{Total} 	&	\textbf{11494} \\
   \hline
   
  \end{tabular}
  \label{table:vocabulary-size-java-annotated}
\end{table}
\end{center}


\fig[scale=0.5]{src/img/entities-distribution-13-categories.png}{img:entities-distribution-13-categories}{Relația între concepte și entități}


\begin{center}
\begin{table}[htb]
  \caption{Dimensiunea vocabularului și numărul de cuvinte ale subsetului adnotat manual din corpus folosind Java}
  \begin{tabular}{|l|r|}
  \hline
   Categorie Entitate & Număr apariții\\
   \hline
  PERSON	&	2549	\\
  LOCATION	&	1007	\\
  MISC	&	942	\\
  ORGANIZATION	&	912	\\
   \hline
   \textbf{Total} 	&	\textbf{5410} \\
   \hline
   
  \end{tabular}
  \label{table:vocabulary-size-java-annotated}
\end{table}
\end{center}


\fig[scale=0.5]{src/img/entities-distribution-4-categories.png}{img:entities-distribution-13-categories}{Relația între concepte și entități}








